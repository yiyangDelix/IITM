\section{Introduction}
5G networks provide dedicated services for URLLC, and the requirements of the International
Telecommunication Union (ITU)~\cite{b24,b14} for URLLC services are defined as follows: the
unidirectional delay from source to destination of a 5G radio access network (RAN) must not
exceed $\si{1}{ms}$, and the delivery success rate must be higher than 99.999\%.
These requirements are extremely demanding for the systems that involve virtualization
management.
The virtualization of operating systems (OS) allow multiple applications,
in an environment sharing the identical host OS, to operate isolatedly.
This has the advantages of fast bootup, easy deployment, low resource consumption and high
operational efficiency\cite{b6}, but also suffers from weak isolation, unstable communication
connections and high network latency\cite{b15}.
The latter prevents the successful performance of URLLC.
This work summarizes the latest sophisticated virtualization methods and makes an examination
of latency optimization methods in order to find virtualization solutions that support the
URLLC service level.

This paper is structured as follows.
In Section 2 the background is presented, while in Section 3 we focus on
comparing container virtualization technology with virtual machine virtualization technology
and summarizing the characteristics of the two OS virtualization technologies.
The latency emerges as measurement latency and performance latency, both should be avoided.
In Section 4, it is about optimizing the latency caused by the measurement process.
The quantifying criteria and prerequisites for measurements are provided as well.
In Section 5, the available and the potential performance latency optimization methods
are overviewed.
Finally, we obtain the conclusion in Section 6 and outline the challenges faced by
virtualization techniques and the outlook for future research.