\section{Conclusion}
In conclusion, the core meaning of virtualization technology is to enable computing
environments to run multiple independent systems simultaneously\cite{b18}.
Both VMs and containers are encapsulated computing environments that
combine various IT components and are independent of the rest of the system.
The main differences between the two are the magnitude, scalability and portability\cite{b15}.
In contrast to the structure of VMs, containers are lightweight, easier to deploy, because they do not
have a separate OS, all containers share the same OS with the host.
This offers the possibility to reduce the latency of virtualization continuously.
However, as an emerging technology, containers are poorly isolated.
Only the CPU can be well isolated, but the memory, disk and network can't be ideally isolated\cite{b8}.
This also presents hidden risks to the reliability of communication.

URLLC has very demanding requirements for low latency.
Therefore, improving accuracy and precision is essential for reliable latency measurements.

From the latency measurement results, the performance of the latency optimization
on bare metal, VMs and containers in the lab's operating environment is satisfactory.
People can optimize the virtualized systems based on VMs by enabling core isolation,
reducing the number of CPU interrupts, as well as disabling virtual cores and energy
saving mechanism, etc.
For the virtualized systems based on containers, people can choose a lightweight container,
optimize container images, use a high-performance storage driver, host networking and resource limits.
With all these mentioned optimization methods, the latency peaks can be significantly reduced even in extreme cases.

The other potentially viable optimization option is a hybrid virtualization approach
that uses a VM as the lightweight host OS, with the upper layer interacting
further with containers.
By leveraging the benefits of each, the performance of communication can be predictably
further improved.